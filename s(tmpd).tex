
\documentclass{article}
\usepackage{amsmath}

\begin{document}

\title{Comparison of Coefficients: \texttt{mgcv} vs. \texttt{glmmTMB}}
\maketitle

\section{Introduction}

When comparing the coefficients in a tmb model and a mgcv model we get this output:
\begin{verbatim}
    > coef(mgcvmod)
(Intercept)   s(tmpd).1   s(tmpd).2   s(tmpd).3   s(tmpd).4   s(tmpd).5   s(tmpd).6   s(tmpd).7   s(tmpd).8   s(tmpd).9 
115.4188502  21.6443900 -35.1723021  -4.9779068  22.7204568  -2.8081798  18.5501319   0.9049146 -64.6723753  18.0106061 
> coef(gtmb1S)
$dummy
     dummy1   dummy2    dummy3    dummy4   dummy5    dummy6   dummy7   dummy8 (Intercept) s(tmpd)1
1 -6.449683 52.03175 0.1844777 -13.80646 41.11658 -26.08403 93.55913 70.78342    115.4189 16.90862
\end{verbatim}

Both the \texttt{mgcv} and \texttt{glmmTMB} models include a smooth term for the variable \( \text{tmpd} \). However, the coefficients and their interpretations differ between the two models due to their different underlying parameterizations. 

\section{Coefficients in \texttt{mgcv}}

In \texttt{mgcv}, the coefficients represent basis expansions for the smooth term \( s(\text{tmpd}) \). There are nine coefficients for this term, labeled from \( s(\text{tmpd}.1) \) to \( s(\text{tmpd}.9) \). These coefficients vary widely in magnitude, suggesting a complex relationship between the response variable and \( \text{tmpd} \).

\section{Coefficients in \texttt{glmmTMB }}

In \texttt{glmmTMB}, the smooth term is treated as a random effect with a specific covariance structure. Consequently, there is only one coefficient for \( s(\text{tmpd}) \), labeled \( s(\text{tmpd}1) \).

\section{Differences and Similarities}

\begin{enumerate}
    \item \textbf{Number of Coefficients}: \texttt{mgcv} provides multiple coefficients for the basis expansion of \( s(\text{tmpd}) \), while \texttt{glmmTMB} provides just one.
    \item \textbf{Intercept}: Both models include an intercept term, and the values are almost identical (\(115.4189\) in \texttt{glmmTMB} and \(115.4188502\) in \texttt{mgcv}).
    \item \textbf{Parameterization}: The coefficients in \texttt{mgcv} are more directly interpretable as basis function weights, whereas in \texttt{glmmTMB} the smooth term is parameterized as a random effect.
    \item \textbf{Scale and Magnitude}: The coefficients in \texttt{mgcv} vary widely, suggesting a more complex relationship with \( \text{tmpd} \). In contrast, \texttt{glmmTMB} provides a single coefficient, due to its different parameterization.
\end{enumerate}

\section{Summary}

While both models aim to capture the relationship between the response variable and \( \text{tmpd} \), they do so in fundamentally different ways due to their parameterizations. Consequently, the coefficients from the two models are not directly comparable. 

\end{document}

