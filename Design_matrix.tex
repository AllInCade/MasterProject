
The function \textbf{s2rPred}, written by Devin Johnson, facilitates the prediction process when using smooth terms represented as type 2 random effects in glmmTMB. It takes a smooth specification, its corresponding random effect representation, and new data to produce prediction matrices that can be used with glmmTMB for making predictions.
\newline


\section{Starting point}

We start off by using the Chicago dataset from the \texttt{gamair} package. It contains air quality and meteorological variables, along with mortality rates. It includes the variables \texttt{tmpd}, \texttt{time}, \texttt{pm10median}, and \texttt{pm25median}, \texttt{so2median} and \texttt{o3median}. The dataset is commonly used for studying the effects of environmental factors on health outcomes.
\newline

We initially used the \texttt{glmmTMB} package to fit a generalized linear mixed model in a standard way. The initial model was specified as follows:

\[
\texttt{death $\sim$ all covariates}
\]

The \texttt{so2median} and \texttt{o3median} variables seem to be insignificant predictors of the response variable \texttt{death}, so we fit this model.

\begin{verbatim}
    tmb_model <- glmmTMB(death ~ so2median + o3median
                 + pm10median + pm25median + 
                   time + tmpd, data = chicago)
\end{verbatim}

This model performs reasonably well.
\newline

We also employed the \texttt{mgcv} package to fit a generalized additive model (spline regression model). The model was specified as:

\begin{verbatim}
    gamm_model <- gamm(death ~s(pm10median) + s(pm25median)
                   + s(time) + s(tmpd), 
                   random = list(time = ~1), data = chicago)
\end{verbatim}
\newline

The \texttt{mgcv} model provided more nuanced insights. The summary of the model indicated the following:


Based on this we decided to attempt fitting a spline regression model using \texttt{glmmTMB} that is of the same form as the \texttt{mgcv} model.
\newline

Here we outline how we incorporate the smooth terms specified by the \textbf{s()} function from \textbf{mgcv} into \textbf{glmmTMB}

\subsection{smooth2random and s2rPred}

We have employed Devin Johnson's \texttt{s2rPred} and Simon Woods \texttt{smooth2random} (\cite{wood2017} function to transform smooth terms into random effects. We also use the \texttt{mgcv} function \texttt{smoothCon} to define a "smooth object", i.e to specify which smooth term to be smoothed in \texttt{mgcv}. It is the \texttt{smooth2random} function which is allowing us to take smooth terms (\textbf{s()} terms) from \texttt{mgcv} and convert them into random effects representation that \texttt{glmmTMB} can handle. A more detailed explanation will follow. Full R code is in \textbf{Appendix}. 



\subsection{Construction of \(X_r\) Terms}

By the method outlined above we construct the following \texttt{glmmTMB} model:

\begin{verbatim}
    gtmb0 <- glmmTMB(formula = death ~ Xr_tmpd + 
                   Xr_time + Xr_pm10 + Xr_pm25, 
                 data = new_data, REML = TRUE)
\end{verbatim}

This model is our attempt at producing a spline regression model in \texttt{glmmTMB}. The \textbf{Xr} terms are supposed to represent smooth terms, like the \textbf{S()} terms in the \texttt{mgcv} model we specified above. 
\newline

But this begs the question, are the \textbf{Xr} terms truly equivalent, or at least similar to the smooth terms in the \texttt{mgcv} model? 

\subsection{Xr terms}

\subsubsection{Type of Smooths for \(X_r\) Terms}

The \(X_r\) terms are derived from the basis functions of the smoother, \(B(x)\), transformed to a random effect parameterization. Mathematically, this can be represented as:

\[
X_r = B(x) \times U \times D
\]

where \(U\) is the transformation matrix and \(D\) is a diagonal matrix containing scaling factors.

\subsubsection{Covariance Matrix}

The covariance matrix, \(\Sigma\), is not diagonal and is given by:

\[
\Sigma = \tau^2 K
\]

where \(K\) is a known matrix derived from the penalty term, and \(\tau^2\) is the variance of the random effects. \(K\) is often based on the second derivative of the smooth function, \(f''(x)\), and is given by:

\[
K = \int B(x)'' B(x)''^T dx
\]

\subsubsection{Type of Penalty}

The penalty term in the likelihood function for the mixed model is:

\[
\text{Penalty} = -\frac{1}{2} \log | \Sigma | - \frac{1}{2} b^T \Sigma^{-1} b
\]

This is not a Ridge penalty, which would be a simple \(L_2\) penalty on the coefficients. Instead, it is a more general form that depends on the second derivative of the smooth function.

\subsubsection{Equivalence of Models}

The models are equivalent in the sense that they both aim to minimize the following objective function:

\[
\text{Objective} = ||y - X\beta - Zb||^2 + \lambda b^T K b
\]

Here, \(y\) is the response, \(X\) is the fixed effects design matrix, \(Z\) is the random effects design matrix, \(\beta\) is the fixed effects vector, \(b\) is the random effects vector, and \(\lambda\) is the smoothing parameter.
\newline

If this seems familiar, it is because this is the \textbf{equivalence of gaussian random effects and quadratically penalized smoothers} which we presented earlier, in action.

\subsubsection{Flexibility of \(X_r\) Terms}

The flexibility of \(X_r\) terms is constrained by the random effects structure. Specifically, the eigen-decomposition of \(K\) will dictate the "wiggliness" of the \(X_r\) terms. In comparison, thin plate regression splines in \texttt{mgcv} have a more flexible basis and are not constrained by a random effects structure.

\subsection{Incorporating Smooth Terms into \texttt{glmmTMB} using \(X_r\) Terms}

The use of \(X_r\) terms could serve as a viable alternative for incorporating smooth term or spline regression functionality into \texttt{glmmTMB}, which currently lacks native support for such features. Below are some points that elaborate on the feasibility and considerations for this approach.

\subsubsection{Advantages}

\begin{enumerate}
    \item \textbf{Model Equivalence}: As discussed earlier, the \(X_r\) terms are mathematically equivalent to the smooth terms in a GAM, under certain conditions. This provides a solid theoretical foundation for their inclusion.
    
    \item \textbf{Flexibility}: The \(X_r\) terms can capture complex relationships in the data, similar to traditional smooth terms, albeit with some limitations.
    
    \item \textbf{Computational Efficiency}: Since \texttt{glmmTMB} is already optimized for handling random effects, the computational burden of adding \(X_r\) terms may be relatively low.
    
    \item \textbf{Generalization}: This approach can be generalized to various types of smoothers and splines, offering a wide range of flexibility in model specification.
\end{enumerate}

\subsubsection{Considerations}

\begin{enumerate}
    \item \textbf{Interpretability}: One of the challenges will be the interpretability of the \(X_r\) terms, especially when compared to traditional smooth terms.
    
    \item \textbf{Overfitting}: Care must be taken to avoid overfitting, which can be a concern with more complex random effects structures.
    
    \item \textbf{User Experience}: The process of converting smooth terms to \(X_r\) terms involves several steps, including basis function selection, transformation, and inclusion as random effects. Automating this process will be crucial for ease of use.
    
    \item \textbf{Software Maintenance}: Implementing this feature would require rigorous testing to ensure that it integrates well with existing \texttt{glmmTMB} functionalities and does not introduce bugs or inconsistencies.
    
    \item \textbf{Computational Complexity}: While \texttt{glmmTMB} is optimized for random effects, the inclusion of \(X_r\) terms could still increase computational time, especially for large datasets or complex models.
    
    \item \textbf{Documentation and Education}: Clear documentation and examples would be essential to help users understand how to properly specify and interpret models with \(X_r\) terms.
\end{enumerate}

In summary, while there are challenges to consider, the use of \(X_r\) terms could be a sound and innovative approach for implementing spline regression functionality in \texttt{glmmTMB}.

