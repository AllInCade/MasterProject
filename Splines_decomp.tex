\documentclass{article}
\usepackage[utf8]{inputenc}
\usepackage{amsmath}
\usepackage{amsfonts}
\usepackage{amssymb}
\usepackage{graphicx}
\usepackage{booktabs}

\title{Differences in spline types}
\author{Erlend Myhre /& Håvard Kolve}
\date{\today}

\begin{document}

\maketitle

\section{Decomposition into 'fixed' and 'random' parts}

In the context of mixed models, spline terms can often be decomposed into fixed and random parts. However, this decomposition is not universal across all spline types. Below, we contrast two popular spline types—cubic regression splines and cubic smoothing splines—to elucidate this point.

\subsection{Cubic Regression Splines}

Cubic regression splines are piecewise cubic polynomials defined between knots. Mathematically, for \(k\) knots, they can be represented as:

\[
f(x) = \beta_0 + \beta_1 x + \sum_{i=1}^{k} \beta_{i+1} (x - \kappa_i)^3_+ 
\]

Here, \(\kappa_i\) are the knots, and \((x - \kappa_i)^3_+\) denotes the truncated power basis. The first two terms, \(\beta_0\) and \(\beta_1 x\), serve as the "fixed" part of the spline, representing a constant and a linear term, respectively. The remaining terms form the "random" part, capturing the non-linear aspects of the data.

\subsection{Cubic Smoothing Splines}

Cubic smoothing splines are defined as the function \(f(x)\) that minimizes:

\[
\sum_{i=1}^{n} (y_i - f(x_i))^2 + \lambda \int [f''(x)]^2 dx
\]

Here, \(y_i\) are the observed data points, \(x_i\) are the corresponding predictor values, and \(\lambda\) is a smoothing parameter. Notice that there is no natural decomposition into a linear (or "fixed") part and a non-linear (or "random") part. The function \(f(x)\) is derived to optimize both the fit to the data and the smoothness of the curve.

\subsubsection{Implications for Implementing Smooth Terms in glmmTMB}

The natural decomposition of some spline types into fixed and random parts has direct implications for their implementation in generalized linear mixed models (GLMMs) using the \texttt{glmmTMB} package in R. Specifically, when using functions like \texttt{smoothCon} and \texttt{smooth2random}, certain spline types will yield both fixed (\(X_f\)) and random (\(X_r\)) effect matrices, while others may yield only a random effect matrix.

\begin{itemize}
    \item \textbf{Cubic Regression Splines}: When using cubic regression splines (`bs="cr"`), both \(X_f\) and \(X_r\) matrices are generated. The \(X_f\) matrix usually captures the unpenalized linear and constant terms, while \(X_r\) represents the penalized cubic terms between the knots. This decomposition aligns well with the mixed model framework, allowing for a seamless implementation in \texttt{glmmTMB}.
    
    \item \textbf{Cubic Smoothing Splines}: For cubic smoothing splines (`bs="cs"`), typically only an \(X_r\) matrix is generated. This is because the spline does not naturally separate into a fixed and a random part, making the entire basis function act as a "random" effect in the model. As a result, when translating this into a mixed model framework using \texttt{glmmTMB}, there is no separate fixed effect component.
\end{itemize}

In summary, the choice of spline type can impact how smoothly it integrates into the \texttt{glmmTMB} framework. Splines that naturally decompose into fixed and random components are easier to implement as they align well with the mixed model structure. On the other hand, splines that do not naturally decompose may require additional considerations or workarounds to fit within this framework.


\section{Penalized B-Splines (P-Splines) in \texttt{mgcv}}

Penalized B-splines, commonly known as P-splines, are a versatile tool for smoothing and are often used for their computational efficiency and simplicity. In \texttt{mgcv}, P-splines are specified using the basis type \texttt{bs="ps"}.

\subsection{Mathematical Representation}

Mathematically, a P-spline can be represented as:

\begin{equation}
f(x) = \sum_{i=1}^{k} \beta_i B_i(x)
\end{equation}

Here, \(B_i(x)\) are the B-spline basis functions, and \(\beta_i\) are the coefficients to be estimated. The B-splines are piecewise polynomial functions defined over a sequence of knots. The degree of the polynomial is commonly 3, leading to cubic B-splines, although other degrees can be used.

\subsection{Penalty Term}

What sets P-splines apart is the penalty term used for smoothing. P-splines commonly use a difference penalty on the coefficients \(\beta\), which can either be a first or second-order finite difference. For first-order differences, the penalty term is:

\begin{equation}
\text{Penalty} = \lambda \sum_{i=2}^{k} (\beta_i - \beta_{i-1})^2
\end{equation}

For second-order differences, the penalty term is:

\begin{equation}
\text{Penalty} = \lambda \sum_{i=3}^{k} (\beta_i - 2\beta_{i-1} + \beta_{i-2})^2
\end{equation}

Here, \(\lambda\) is the smoothing parameter that controls the strength of the penalty. Higher values of \(\lambda\) result in a smoother curve, while lower values allow for more wiggly fits.



\subsection{Summary}

P-splines combine the flexibility of B-splines with the simplicity of a difference penalty. They are computationally efficient, easy to implement, and can be very effective for a wide range of smoothing tasks. They are particularly useful when a balance of flexibility and smoothness in the estimated function \(f(x)\) is desired.


\section{Thin plate splines:}

In the context of generalized linear mixed models (GLMMs) implemented via \texttt{glmmTMB} and using machinery from \texttt{mgcv}, the decomposition of a spline into \(Xf\) and \(Xr\) terms captures the fixed and random effects components of the smooth term, respectively.

\subsection{Thin Plate Splines and Their Decomposition}

\begin{enumerate}
    \item \textbf{Full Thin Plate Spline (\texttt{bs="tp"})}:
    \begin{itemize}
        \item In a full thin plate spline, the basis functions are defined in such a way that they apply across the entire range of the data. They are more global in nature.
        \item When using the \texttt{smoothCon} and \texttt{smooth2random} functions to prepare these splines for \texttt{glmmTMB}, the decomposition usually yields only an \(Xr\) term, with no corresponding \(Xf\). This is because the penalty for thin plate splines makes it difficult to naturally separate them into fixed and random components.
        \item In the context of GLMMs, the entire spline term will behave as a random effect in the model.
    \end{itemize}
    
    \item \textbf{Thin Plate Regression Splines (\texttt{bs="tp"}, \(k=<\text{some integer}>\))}:
    \begin{itemize}
        \item Similar to full thin plate splines, but they are a lower-rank approximation and hence computationally more efficient.
        \item The decomposition into \(Xf\) and \(Xr\) would depend on the specific properties of the basis functions and the penalty term. It's likely that you will also only get an \(Xr\) term.
    \end{itemize}
\end{enumerate}


\subsection{Mathematical Representation of Thin Plate Regression Splines}

Thin plate regression splines are a computationally efficient approximation of full thin plate splines. They are generally represented using a subset of the basis functions used in full thin plate splines, effectively creating a lower-rank approximation. The mathematical representation can be written as:

\[
f(x) = \beta_0 + \sum_{i=1}^{m} \beta_i x_i + \sum_{j=1}^{k} \gamma_j \phi(|| x - x_{j} ||)
\]

Here, \( \beta_0 \) is the intercept, \( \beta_i \) are the coefficients for the linear terms \( x_i \), \( \gamma_j \) are the coefficients for the radial basis functions \( \phi(|| x - x_{j} ||) \), \( x_j \) are the selected knot locations, and \( k \) is the reduced rank of the spline. The radial basis function \( \phi(r) \) depends on the dimensionality and is usually more complex in higher dimensions.

This lower-rank approximation makes thin plate regression splines computationally more efficient while still retaining many desirable properties of full thin plate splines.


\subsubsection{Mathematical Representation of Full Thin Plate Splines}

Full thin plate splines are a type of global smoother that use radial basis functions to represent the smooth. In the one-dimensional case, the mathematical representation of a full thin plate spline can be written as:

\[
f(x) = \sum_{i=1}^{n} \alpha_i \phi(|| x - x_i ||)
\]

Here, \( \phi(r) \) is a radial basis function, \( \alpha_i \) are the coefficients to be estimated, \( x_i \) are the knot locations, and \( n \) is the number of knots. The radial basis function \( \phi(r) \) is usually defined as \( r^2 \log(r) \) for \( r > 0 \) and zero otherwise. This function has desirable mathematical properties, such as rotational invariance, which makes it useful for multidimensional smoothing as well.

The complexity and global nature of these basis functions make full thin plate splines computationally intensive, especially when the number of knots is large or the dimensionality is high.




\subsection{Summary}

Thin plate splines are more likely to be represented entirely as random effects (\(Xr\)) in the mixed model framework when prepared for \texttt{glmmTMB} using \texttt{mgcv}'s \texttt{smoothCon} and \texttt{smooth2random}. The global nature of these splines and their penalty structure make it difficult to decompose them into separate fixed (\(Xf\)) and random (\(Xr\)) components. Thin plate regression splines however will have both components added to their model, as we see the fixed terms as the beta-values in its mathematical representation.


\section{Cyclical Cubic Splines and Comparison with Regular Cubic Splines}

\subsection{Overview of Cyclical Cubic Splines}

The cyclical cubic spline is a special case of cubic splines where the function is constrained to be cyclic at the endpoints. This means that the function value and its first and second derivatives at the endpoints are the same, ensuring smooth transitions.

\subsubsection{Mathematical Representation}

Mathematically, a cyclical cubic spline \( f(x) \) can be represented as a linear combination of cubic basis functions \( B_i(x) \) and coefficients \( \beta_i \), similar to regular cubic splines:

\[
f(x) = \sum_{i=1}^{k} \beta_i B_i(x)
\]

Here, \( k \) is the number of knots, and \( B_i(x) \) are the cubic basis functions. However, the basis functions are constructed to ensure that the function is cyclic. Specifically, they are designed so that:

\[
f(x_1) = f(x_n), \quad f'(x_1) = f'(x_n), \quad f''(x_1) = f''(x_n)
\]

\subsubsection{Penalty Term}

The penalty term for cyclical cubic splines is usually based on the integrated squared second derivative, similar to regular cubic splines:

\[
\text{Penalty} = \lambda \int [f''(x)]^2 \, dx
\]

However, the penalty matrix would be constructed to reflect the cyclical nature of the spline, taking into account the constraints at the endpoints.


\subsubsection{Summary}

Cyclical cubic splines are a specific type of cubic spline designed for modeling periodic or seasonal data. They are constructed to be cyclic at the endpoints and are typically represented entirely as random effects when used in the context of GLMMs, especially when prepared for use in \texttt{glmmTMB} via \texttt{mgcv}.

\subsection{Comparison of Cubic Smoothing Splines and Cubic Cyclical Splines}

\paragraph{Cubic Smoothing Splines (\texttt{bs="cs"})}

Cubic smoothing splines are designed to model data without any specific constraints at the endpoints. These splines are often smoother due to a different penalty term that targets the entire curve. When decomposed for use in \texttt{glmmTMB} through \texttt{mgcv}'s \texttt{smoothCon} and \texttt{smooth2random} functions, cubic smoothing splines typically do not decompose into a fixed (\(X_f\)) component. This is because the penalty term for smoothing splines makes it difficult to naturally separate them into fixed and random components.

\paragraph{Cubic Cyclical Splines (\texttt{bs="cc"})}

Cubic cyclical splines are constrained to have the same function value at both endpoints. They are useful for modeling cyclical or seasonal data. This cyclical constraint often leads to a single \(X_f\) term that captures the constant cyclical effect across the data range. The \(X_f\) term essentially represents a fixed effect to ensure the function is smooth and equal at both ends.



\subparagraph{Summary}

The presence or absence of \(X_f\) and \(X_r\) terms in the decomposition into fixed and random effects components can depend on the specific constraints and penalties associated with the spline type. Cubic smoothing splines (\texttt{bs="cs"}) are often so smooth that they do not naturally decompose into \(X_f\) and \(X_r\) terms. In contrast, cubic cyclical splines (\texttt{bs="cc"}) have a cyclic constraint that often leads to a single \(X_f\) term.


\section{Basis functions in glmmTMB}

\begin{table}[h]
\centering
\caption{Compatibility of Spline Types in \texttt{glmmTMB}}
\begin{tabular}{lccc}
\toprule
Spline & bs & Compatible & Reasoning \\
\midrule
Cubic Regression & \texttt{"cr"} & \checkmark &  \\
Cyclical Cubic & \texttt{"cc"} & $\times$ & Only one Xf-term \\
Cubic Smoothing & \texttt{"cs"} & $\times$ &  No Xf-terms \\
Thin Plate Regression & \texttt{"tp"} & \checkmark &   \\
B-splines & \texttt{"bs"} &  \checkmark &  \\
P-splines & \texttt{"ps"} &  \checkmark &  \\
Two-dimensional Tensor Product & \texttt{"t2"} & \checkmark & \\
Two-dimensional Tensor Product & \texttt{"te"} & $\times$ & Not supported by smooth2random (gamm4) \\
Shrinkage Smooth & \texttt{"fs"}  & \checkmark & \\
Adaptive & \texttt{"ad"}  &$\times$ & Not supported by smooth2random  \\
Random Effect & \texttt{"re"}  & $\times$ & No Xf-terms \\
% Add other splines as needed
\bottomrule
\end{tabular}
\end{table}


\end{document}