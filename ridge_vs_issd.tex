\documentclass[12pt, twoside,hidelinks]{article}
\usepackage[utf8]{inputenc}
\usepackage[english]{babel}
\usepackage{graphicx}
\usepackage{placeins}
\usepackage{float} % For precise figure placement, use the float package
\usepackage{multirow}
\usepackage{tabularx}
\usepackage{subcaption}
% For mathematical typesetting
\usepackage{amsmath, amssymb, amsthm}				\usepackage{natbib}									\usepackage{pgfplots}		
\usepackage{booktabs}
\usepackage{amssymb} % For checkmark and cross
\usepackage{enumitem}
\setlist[itemize]{itemsep=0pt, parsep=0pt, leftmargin=*}
\usepackage{listings}												  
\usepackage{mathtools}												   			\usepackage{longtable}
\usepackage{tikz}
\usetikzlibrary{shapes.geometric, arrows, positioning, chains}

% Define block styles
\tikzstyle{startstop} = [rectangle, rounded corners, minimum width=3cm, minimum height=1cm,text centered, draw=black, fill=red!30]
\tikzstyle{io} = [trapezium, trapezium left angle=70, trapezium right angle=110, minimum width=3cm, minimum height=1cm, text centered, draw=black, fill=blue!30]
\tikzstyle{process} = [rectangle, minimum width=3cm, minimum height=1cm, text centered, draw=black, fill=orange!30]
\tikzstyle{arrow} = [thick,->,>=stealth]

\usepackage{smartdiagram}
\usesmartdiagramlibrary{additions}
\usepackage{lipsum}

\usepackage{placeins}

% For setting margins
\usepackage[top=3cm, bottom=3cm, inner=2.5cm, outer=2.5cm, marginparwidth=4cm]{geometry}				

\usepackage{hyperref}													 			 																																	 % For hyperlinks
\usepackage{booktabs}															   																																   	   % For better quality of tables

\usepackage{enumitem}

\usepackage{float}														\usepackage{wrapfig}	

\usepackage{upgreek}																																				% Help with objects that doesn't fit in the present page

\usepackage{subcaption}	
% setting exercise style
\theoremstyle{definition}
\newtheorem{oppgave}{Task}[section]
\newtheorem{definition}{Definition}
\newtheorem{theorem}{Theorem}[section]
\newtheorem{Lemma}{Lemma}[section]
\numberwithin{equation}{section}





\begin{document}



\section{Ridge Penalty vs ISSD Penalty}

In the realm of statistical modeling, particularly when dealing with Generalized Additive Models (GAMs), a crucial consideration is the choice of smoothing or regularization technique. The Ridge penalty and Integrated Squared Second Derivative (ISSD) penalty serve similar purposes in mitigating overfitting, but their appropriateness can vary based on the nature of the data and the model's characteristics. The Ridge penalty, imposing a uniform shrinkage, is often preferred in scenarios with multicollinearity and when a global smoothing effect is desired. On the other hand, the ISSD penalty, focusing on the smoothness of the function by penalizing the curvature, is typically favored for data that inherently follow a smooth underlying trend. Choosing between these penalties is a strategic decision that can significantly influence a model's performance and interpretability.

\begin{table}[ht]
\centering
\begin{tabular}{p{0.45\textwidth}p{0.45\textwidth}}
\hline
\textbf{Ridge Penalty Characteristics} & \textbf{ISSD Penalty Characteristics} \\
\hline
Preferred for data with multicollinearity & Suited for data with a naturally smooth underlying structure \\
Effective in high-dimensional settings where the number of predictors exceeds the number of observations & Ideal for evenly distributed data without abrupt changes \\
Uniformly shrinks coefficients, addressing overfitting effectively & Smoothens the function by penalizing high curvature, leading to a locally adaptive fit \\
Less sensitive to outliers, providing stable solutions & Can adapt to a specific level of smoothness, beneficial for data with consistent variability \\
Handles global structure in the data, making it suitable for complex models with multiple predictors & Preserves the interpretability of the model by maintaining the functional form \\
\hline
\end{tabular}
\caption{Comparison of data and model characteristics that are conducive to the use of Ridge versus ISSD penalties in regularization.}
\label{table:ridge_vs_issd}
\end{table}


\end{document}